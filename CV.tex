%% start of file `template.tex'.
%% Copyright 2006-2013 Xavier Danaux (xdanaux@gmail.com).
%
% This work may be distributed and/or modified under the
% conditions of the LaTeX Project Public License version 1.3c,
% available at http://www.latex-project.org/lppl/.


\documentclass[12pt,a4paper,roman]{moderncv}        % possible options include font size ('10pt', '11pt' and '12pt'), paper size ('a4paper', 'letterpaper', 'a5paper', 'legalpaper', 'executivepaper' and 'landscape') and font family ('sans' and 'roman')

% modern themes
\moderncvstyle{banking}                            % style options are 'casual' (default), 'classic', 'oldstyle' and 'banking'
\moderncvcolor{blue}                                % color options 'blue' (default), 'orange', 'green', 'red', 'purple', 'grey' and 'black'
%\renewcommand{\familydefault}{\sfdefault}         % to set the default font; use '\sfdefault' for the default sans serif font, '\rmdefault' for the default roman one, or any tex font name
%\nopagenumbers{}                                  % uncomment to suppress automatic page numbering for CVs longer than one page

% character encoding
\usepackage[utf8]{inputenc}                       % if you are not using xelatex ou lualatex, replace by the encoding you are using
%\usepackage{CJKutf8}                              % if you need to use CJK to typeset your resume in Chinese, Japanese or Korean

% adjust the page margins
\usepackage[scale=0.75]{geometry}
%\setlength{\hintscolumnwidth}{3cm}                % if you want to change the width of the column with the dates
%\setlength{\makecvtitlenamewidth}{10cm}           % for the 'classic' style, if you want to force the width allocated to your name and avoid line breaks. be careful though, the length is normally calculated to avoid any overlap with your personal info; use this at your own typographical risks...

\usepackage{import}

% personal data
\name{Rafael Alberto}{Rivera-Soto}
% \title{Curriculum Vitae}                               % optional, remove / comment the line if not wanted
\address{5255 Norma Way, Apt. 126, Livermore CA 94550}{}{}% optional, remove / comment the line if not wanted; the "postcode city" and and "country" arguments can be omitted or provided empty
\phone[mobile]{+787 220 2975}                   % optional, remove / comment the line if not wanted
\phone[fixed]{925 423 4490}                    % optional, remove / comment the line if not wanted
%\phone[fax]{+3~(456)~789~012}                      % optional, remove / comment the line if not wanted
\email{riverasoto1@llnl.gov}                               % optional, remove / comment the line if not wanted
\homepage{www.github.com/rrivera1849}                         % optional, remove / comment the line if not wanted
%\extrainfo{additional information}                 % optional, remove / comment the line if not wanted
%\photo[64pt][0.4pt]{picture}                       % optional, remove / comment the line if not wanted; '64pt' is the height the picture must be resized to, 0.4pt is the thickness of the frame around it (put it to 0pt for no frame) and 'picture' is the name of the picture file
%\quote{Some quote}                                 % optional, remove / comment the line if not wanted

% to show numerical labels in the bibliography (default is to show no labels); only useful if you make citations in your resume
\makeatletter
\renewcommand*{\bibliographyitemlabel}{\@biblabel{\arabic{enumiv}}}
%\makeatother
%\renewcommand*{\bibliographyitemlabel}{[\arabic{enumiv}]}% CONSIDER REPLACING THE ABOVE BY THIS

% bibliography with mutiple entries
%\usepackage{multibib}
%\newcites{book,misc}{{Books},{Others}}
%----------------------------------------------------------------------------------
%            content
%----------------------------------------------------------------------------------
\begin{document}
%\begin{CJK*}{UTF8}{gbsn}                          % to typeset your resume in Chinese using CJK
%-----       resume       ---------------------------------------------------------
\makecvtitle

\small{Undergraduate researcher at Lawrence Livermore National Laboratory. Passionate about the possibilities between the intersection between Artificial Intelligence and Computer Security.}

\section{Employment}

\cventry{September 2015--Present}{Computer Scientist}{Lawrence Livermore National Laboratory}{Livermore,CA}{}{}
\vspace{5pt}

As a Computer Scientist in LLNL, I've worked on a wide variety of efforts some of which include but are not limited to:
\begin{itemize}
	\item Created models for a coupled transmission-communication simulation for studying the effects of cyber attacks on the transmission system. 
	
	\item Built a neural network model for predicting the amount of PV generation in a particular distribution installation. Results had to meet 
	
	Built a NARX neural network model for predicting the amount of PV generation in a particular installation.
	
	\item Created a convolutional neural network model for identifying the compilers, versions and flags used to create a binary.  

	\item Researching neural network models to identify the author of a particular source code sample. 
\end{itemize}

\section{Education}

\subsection{Academic Qualifications}

\cventry{2016--Present}{SCPD Student}{Stanford University}{Stanford, C.A.}{}{}

\vspace{2pt}

\cventry{2012--2015}{Bachelor of Science, Computer Engineering}{Universidad del Turabo}{Gurabo, Puerto Rico}{\textit{Cumulative GPA 3.78}}{}

\vspace{2pt}

\cventry{August 2009--May 2012}{High School}{Thomas Alva Edison School}{Caguas, Puerto Rico}{}{}

\section{Experience}

\subsection{Internships}

\cventry{June 2015--August 2015}{Undergraduate Intern--Cyber Defenders Student Program}
{Lawrence Livermore National Laboratory}{Livermore, CA}{}{}

\begin{itemize}
	\item Created an authentication system for an internal web application. The system is able to account for access from three different security classification networks and adjust accordingly. 
\end{itemize}

\vspace{5pt}

\cventry{June 2014--August 2014}{Undergraduate Intern--Cyber Defenders Student Program}
{Lawrence Livermore National Laboratory}{Livermore, CA}{}{}

\begin{itemize}
    \item Designed a model which describes the amount of time it takes a power grid network to recover to against a certain amount of damage.

	\item Created a simulation to study the robustness and resilience of the power grid against various attacks.
\end{itemize}

\vspace{5pt}

\cventry{June 2013--August 2013}{Undergraduate Intern--Cyber Defenders Student Program}
{Lawrence Livermore National Laboratory}{Livermore, CA}{}{}

\begin{itemize}
	\item Designed a model which describes the effect of cascading power failures in a power grid network.
	
	\item Implemented a simulation to study the resilience of different network models against various types of attacks.
\end{itemize}

\subsection{Leadership}

\begin{itemize}
	\item Founding board member of the ACM (Association for Computing Machinery) and Tau Alpha Omega chapters at the University of Turabo.
	
	\item Organized student workshops and documented reunions for the Association for Computing Machinery.
\end{itemize}

\section{Achievements}

\cventry{March 2014--Present}{Member of the Tau Alpha Omega Engineering Honor Society}{Universidad del Turabo}{Gurabo, Puerto Rico}{}{}

\vspace{2pt}

\cventry{August 2012--August 2014}{Caribbean Computer Center of Excellence scholar}{Universidad del Turabo}{Gurabo, Puerto Rico}{}{}

\vspace{2pt}

\cventry{March 7, 2014}{Presenter at Puerto Rico Researchers Council}{AGMUS Research Symposium}{San Juan, Puerto Rico}{}{Presented results of compressing hyper-spectral images using neural networks.}

\vspace{2pt}

\cventry{November 9, 2013}{Participant at Caribbean Finals, ACM-ICPC}{Insituto Tecnologico de Santo Domingo}{Dominican Republic}{}{}

\vspace{2pt}

\cventry{October 5, 2013}{Second place in the ACM-ICPC Puerto Rico National Competitions}{University of Puerto Rico}{Bayamon, Puerto Rico}{}{}

\vspace{2pt}

\cventry{April 27, 2013}{Second place in the UPR-Bayamon Computer Programming Competition}{University of Puerto Rico}{Bayamon, Puerto Rico}{}{}

\section{Projects}

\textbf{Machine Learning for Brand Loyalty Prediction}\\
\cvlistitem{Using data from the Nielsen Consumer Panel Dataset, developed a brand loyalty prediction system}
\cvlistitem{Implemented three machine learning techniques: SVM, Multinomial Logistic Regression and K-Means Clustering on consumer data}
\cvlistitem{The model was able to predict whether or not a user would be loyal to the brand with an accuracy of 95\%.}


\vspace{5pt}

\textbf{Machine Learning for Conversational Agents}\\
\cvlistitem{Used data from the Ubuntu Dialog Corpus as well as Twitter to create a retrieval-based and generative conversational agent}
\cvlistitem{Implemented a retrieval-based agent using two LSTM's to encode the question and candidate answers, thereby comparing the similarity between the two and giving it a score}
\cvlistitem{Implemented a Sequence to Sequence model with an attention mechanism to generate responses to arbitrary conversation context}

\vspace{5pt}

\textbf{Algorithm Implementations}\\
\cvlistitem{Implemented various algorithms: Tries, Merge Sort, Quick Sort and contributed them to an open source project: \url{https://github.com/kennyledet/Algorithm-Implementations}}

\section{Skills}

\textbf{Programming Languages:} Proficient in Python, C++ Strong Experience in C and Java
\vspace{2pt}

\textbf{Environment} Windows, Linux
\vspace{2pt}
	
\textbf{Deep Learning Libraries:} PyTorch, Tensorflow, Keras
\vspace{2pt}
	
\textbf{Computer Forensic Tools:} IDA, OllyDbg, Autopsy
\vspace{2pt}
	
Fluent Spanish and English speaker
\vspace{2pt}

% Publications from a BibTeX file without multibib
%  for numerical labels: \renewcommand{\bibliographyitemlabel}{\@biblabel{\arabic{enumiv}}}% CONSIDER MERGING WITH PREAMBLE PART
%  to redefine the heading string ("Publications"): \renewcommand{\refname}{Articles}
\nocite{*}
\bibliographystyle{abbrv}
\bibliography{publications}                        % 'publications' is the name of a BibTeX file

% Publications from a BibTeX file using the multibib package
%\section{Publications}
%\nocitebook{book1,book2}
%\bibliographystylebook{plain}
%\bibliographybook{publications}                   % 'publications' is the name of a BibTeX file
%\nocitemisc{misc1,misc2,misc3}
%\bibliographystylemisc{plain}
%\bibliographymisc{publications}                   % 'publications' is the name of a BibTeX file

%-----       letter       ---------------------------------------------------------

\end{document}


%% end of file `template.tex'.
